\section{Parser}
\label{sec:parser}
The parser is the first component, that any SQL statement encounters.
It consists of the parser logic, as well as a scanner (\autoref{sec:scanner}), which is responsible for splitting the text into actual tokens.

For simplicity reasons, and a special feature of the Go runtime, we decided to not follow the usual, quite heavyweight approach for a parser, which we showcase in \autoref{fig:heavyweight_parser_arch}.
Instead, we use the approach showcased in \autoref{fig:lightweight_parser_arch}.
We discard the interpreter, because we decided to use the parse tree as an input to a compiler, which then generates an intermediary representation (\autoref{sec:intermediary_representation}), that the engine can execute.

\begin{figure}
    \centering
    \begin{subfigure}{.5\textwidth}
        \centering
        \begin{tikzpicture}[node distance = 3em, auto]
        % nodes
        \node [cloud] (input) {SQL input};
        \node [block, below of=input] (decode) {decoder};
        \node [block, below of=decode] (tokenize) {tokenizer};
        \node [block, below of=tokenize] (parse) {parser};
        \node [cloud, below of=parse] (parsetree) {parse tree};
        \node [block, below of=parsetree] (interpret) {interpreter};
        \node [cloud, below of=interpret] (ast) {AST};
        % edges
        \path [line] (input) -- (decode);
        \path [line] (decode) -- (tokenize);
        \path [line] (tokenize) -- (parse);
        \path [line] (parse) -- (parsetree);
        \path [line] (parsetree) -- (interpret);
        \path [line] (interpret) -- (ast);
        \end{tikzpicture}
        \caption{Heavyweight parser architecture}
        \label{fig:heavyweight_parser_arch}
    \end{subfigure}%
    \begin{subfigure}{.5\textwidth}
        \centering
        \begin{tikzpicture}[node distance = 3em, auto]
        % nodes
        \node [cloud] (input) {SQL input};
        \node [block, below of=input] (tokenize) {tokenizer};
        \node [block, below of=tokenize] (parse) {parser};
        \node [cloud, below of=parse] (parsetree) {parse tree};
        % edges
        \path [line] (input) -- (tokenize);
        \path [line] (tokenize) -- (parse);
        \path [line] (parse) -- (parsetree);
        \end{tikzpicture}
        \caption{Lightweight parser architecture}
        \label{fig:lightweight_parser_arch}
    \end{subfigure}
    \caption{Both parser architectures}
    \label{fig:parser_archs}
\end{figure}

Despite it is shown in \autoref{fig:lightweight_parser_arch}, we don't really skip the decoder, but it is not an actual component in the database.
The Go runtime can convert \mintinline{go}{[]byte} input into \mintinline{go}{[]rune}, which are UTF-8 characters.
If the input string is not a valid UTF-8 string, then the input will not be converted at all.
An example for an invalid UTF-8 string would be \mintinline{go}{"hello\xf3world"}, whereas \mintinline{go}{"hello\u00f3world"} would be very valid.

For parsing, we use the parser, which produces a parse tree.
For simplicity reasons, we call that parse tree \textit{AST} (Abstract Syntax Tree) in our implementation.

\subsection{Scanner}
\label{sec:scanner}